\section{Gestione di Progetto}

\subsection{Suddivisione del lavoro progettuale}
Il progetto è stato iniziato all'inizio del mese di giugno e si è concluso il giorno precedente alla consegna (4 luglio 2020).\\
Il progetto, nella la sua completezza ha richiesto circa \textbf{110} ore equalmente divise tra i due componenti del gruppo, che sono:
\begin{itemize}
    \item Buratto Enrico - 1142644
    \item Ostanello Emanuele - 1143439
\end{itemize}
Entrambi i componenti hanno contribuito a tutte le parti del progetto; la progettazione iniziale dei diversi componenti del progetto è stata concepita in totale collaborazione tramite la piattaforma di comunicazione vocale \textit{Discord}, necessaria nel periodo di distanziamento sociale che si è sovrapposto a buona parte del periodo di progettazione e sviluppo. Tuttavia, i singoli componenti hanno approfondito maggiormente un determinato ambito di progetto; nello specifico:
\begin{table}[H]
\begin{center}
\begin{tabular}{c|c}
\textbf{Enrico Buratto}        & \textbf{Emanuele Ostanello} \\ \hline
Codifica Gerarchia e Container & Apprendimento Qt            \\
Codifica Model                 & Codifica View              
\end{tabular}
\end{center}
\end{table}

\subsection{Timeline individuale}
Il lavoro individuale ha richiesto circa \textbf{55 ore} sulle 50 ore previste; le ore in eccesso sono dovute principalmente alla difficoltà di comunicazione a distanza, dovuta al periodo attuale. Nonostante questo, si può dire che il carico di lavoro è stato ben calibrato in base alla disponibilità oraria personale e a quanto sviluppato.
\begin{table}[H]
\begin{center}
\begin{tabular}{l|c}
\textbf{Fasi progettuali}        & \textbf{Ore utilizzate} \\ \hline
Analisi del problema             & 5                       \\
Progettazione                    & 13                      \\
Apprendimento Qt                 & 4                       \\
Codifica e implementazione Model & 17                      \\
Codifica e implementazione View  & 4                       \\
Test e Debug                     & 8                       \\
Relazione                        & 4                       \\ \hline
\textbf{Ore totali}              & \textbf{55}            
\end{tabular}
\end{center}
\end{table}

\subsection{Ambiente di lavoro}
Il progetto è stato sviluppato con \texti{Qt Creator v4.11.2} e \textif{Atom v1.47.0} su sistema operativo \textit{Arch Linux}; il compilatore usato è stato \textit{gcc v10.1.0}. Essendo il compilatore e la versione di \textit{Qt} del sistema di sviluppo più aggiornati rispetto alle specifiche, sono stati effettuati frequenti allineamenti con la macchina virtuale, contenente \textit{Ubuntu 18.04 LTS} con \textit{gcc v7.4.0} e \textit{Qt} alla versione v5.9.5. Questo ha permesso di verificare la compatibilità di quanto prodotto con le specifiche tecniche richieste.

\subsection{Istruzioni di compilazione ed esecuzione}
Il progetto prevede la compilazione tramite file .pro e tool qmake. Le istruzioni per la compilazione e l'esecuzione sono quindi le seguenti:
\begin{center}
\centering
\begin{verbatim}
  $ qmake CrownSouls.pro
  $ make
  $ ./CrownSouls
\end{verbatim}
\end{center}

Viene inoltre fornito un file \texttt{.xml}, locato nella cartella \texttt{extra}, contenente un inventario precompilato di prova.

\subsection{Considerazioni finali}
Il progetto ha richiesto un modesto impegno, soprattutto per quanto riguarda la progettazione e l'organizzazione del lavoro, avvenute a distanza. Una volta che il gruppo si è organizzato e il lavoro è stato diviso, però, lo sviluppo è stato abbastanza lineare, costante nel tempo e senza particolari complicazioni. \\
Entrambi i componenti del gruppo hanno utilizzato una parte considerevole di tempo per l'apprendimento e l'implementazione del \textit{proxy model}, che inizialmente ha dato non pochi problemi. Una volta risolti questi, però, lo sviluppo è continuato velocemente. \\
Concludendo, possiamo affermare che il progetto è stato abbastanza lungo e impegnativo, ma lo studio individuale e la scrittura di codice "reale" ha portato i suoi frutti.
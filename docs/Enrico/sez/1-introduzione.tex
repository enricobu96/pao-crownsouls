\section{Introduzione}
\subsection{Abstract}
Si vuole realizzare un programma per la gestione di un inventario giocatore per il gioco \textit{Action RPG} \textfb{"Dark Souls"}. Il giocatore possiede svariati elementi di diversi tipi; questi elementi possono infatti essere:
\begin{itemize}
  \begin{description}
    \item \textbf{Armature}: oggetti indossati dal giocatore per aumentare la resistenza al danno inflitto dai nemici; \\
    \item \textbf{Armi}: oggetti utilizzati dal giocatore per infliggere danno ai nemici; \\
    \item \textbf{Anelli}: oggetti utili al giocatore per l'aumento delle proprie statistiche; una volta indossati nel gioco, il giocatore vedrà aumentate alcune statistiche personali o delle armi;
    \item \textbf{Scudi}: oggetti utilizzati dal giocatore per ridurre il danno dai colpi nemici;
    \item \textbf{Guanti}: oggetti utilizzati sia come armatura, poiché aumentano la resistenza al danno, sia come arma, poiché permettono di infliggere danno;
    \item \textbf{Scudi d'attacco}: oggetti utilizzati sia come scudo, poiché riducono il danno dai colpi nemici, sia come arma, poiché permettono di infliggere danno.
  \end{description}
\end{itemize}

Un inventario è composto da un insieme di oggetti appartenenti alle diverse tipologie; ogni oggetto presente nell'inventario possiede delle caratteristiche tecniche proprie della categoria di appartenenza. \\
Il programma deve poter simulare un inventario di questo tipo, permettendo l'inserimento, la rimozione e la visualizzazione degli oggetti e delle loro proprietà.


\subsection{Funzionalità}
Per facilitare la visualizzazione degli oggetti dell'inventario, questi sono suddivisi all'interno del programma in quattro diverse schede; ogni scheda rappresenta una sottosezione dell'inventario, e mostra al suo interno solo gli elementi appartenenti alla categoria indicata dal titolo. Per la gestione degli oggetti dell'inventario sono presenti le seguenti funzionalità:

\begin{itemize}
  \item Caricamento ed esportazione dell'intero inventario da e su file XML;
  \item Aggiunta di un nuovo oggetto all'inventario;
  \item Modifica di un elemento già presente nell'inventario;
  \item Rimozione di un elemento dell'inventario;
  \item Rimozione di tutti gli elementi presenti;
  \item Visualizzazione di oggetti dell'inventario divisi per categoria di appartenenza;
  \item Visualizzazione delle caratteristiche di ogni elemento dell'inventario, comprese alcune statistiche calcolate automaticamente dal programma;
  \item Visualizzazione di avvisi d'errore.
\end{itemize}
